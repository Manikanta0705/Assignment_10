\documentclass{beamer}
\usetheme{CambridgeUS}



\usepackage{enumitem}
\usepackage{tfrupee}
\usepackage{amsmath}
\usepackage{amssymb}
\usepackage{gensymb}
\usepackage{graphicx}
\usepackage{txfonts}

\def\inputGnumericTable{}

\usepackage[latin1]{inputenc}                                 
\usepackage{color}                                            
\usepackage{array}                                            
\usepackage{longtable}                                        
\usepackage{calc}                                             
\usepackage{multirow}                                         
\usepackage{hhline}                                           
\usepackage{ifthen}
\usepackage{caption} 
\captionsetup[table]{skip=3pt}  
\providecommand{\pr}[1]{\ensuremath{\Pr\left(#1\right)}}
\providecommand{\cbrak}[1]{\ensuremath{\left\{#1\right\}}}
\renewcommand{\thefigure}{\arabic{table}}
\renewcommand{\thetable}{\arabic{table}}                                     

\title{AI1110 \\ Assignment 10}
\author{MANIKANTA UPPULAPU \\ (BT05)}
\date{\today}


\begin{document}
	\begin{frame}
		\titlepage
	\end{frame}
	\begin{frame}{Outline}
		\tableofcontents
	\end{frame}
	\section{Question}
	\begin{frame}{Problem Statement}
	\begin{block}
{Question:}	 Let x and y be independent gamma random variables with parameters (\alpha_1,\beta) and (\alpha_2,\beta) respectively\\
(a) Determine the p.d.f.s of the random variables x + y and $\frac{x}{x+y}$\\
(b) Show that x + y and $\frac{x}{y}$ are independent random variables\\
(c) Show that x + y and $\frac{x}{x+y}$ are independent gamma and beta random variables, respectively.
\end{block}
\end{frame}
	\section{Solution}
	\begin{frame}{Solution}
(a) let z = x + y and w = $\frac{x}{x+y}$\\
f_{xy}(x,y) = f_x(x) f_y(y) = $$ \frac{1}{\alpha ^{\alpha_1+\alpha_2}\Gamma_{(\alpha_1)}\Gamma_{(\alpha_2)}}x^{\alpha_1-1}y^{\alpha_2-1}e^{\frac{-(x+y)}{\beta}}$$  &  x>0, y>0\\
note that $0<z<1$, since x and y are non-negative random variables 
\begin{align}
F_z(Z) &= P(z\leq Z) = P\left(\frac{x}{x+y} \leq z\right) = P\left(x \leq y\frac{z}{1-z}\right)\\
&=\int_{0}^{\infty} \int_{0}^{\frac{yz}{1-z}} f_{xy}(x,y)dx.dy
\end{align}
differentiation with respect to z gives
\begin{align}
f_z(Z) &= \int_{0}^{\infty} \frac{y}{(1-z)^{2}} f_{xy}\left(\frac{yz}{1-z},y\right)dy
\end{align}
\end{frame}

\begin{frame}{Solution}
\begin{align}
&= \int_{0}^{\infty} \frac{y}{(1-z)^{2}} \frac{1}{\alpha ^{\alpha_1+\alpha_2}\Gamma_{(\alpha_1)}\Gamma_{(\alpha_2)}}\left(\frac{yz}{1-z}\right)^{\alpha_1-1} y^{\alpha_2-1} e^{\frac{-y}{(1-z)\alpha}}dy\\
&=  \frac{1}{\alpha ^{\alpha_1+\alpha_2}\Gamma_{(\alpha_1)}\Gamma_{(\alpha_2)}} \frac{z^{\alpha_1-1}}{(1-z)^{\alpha_1+1}} \int_{0}^{\infty} y^{\alpha_1+\alpha_2-1} e^{\frac{-y}{\alpha(1-z)}}dy\\
&= \frac{z^{\alpha_1-1}(1-z)^{\alpha_2-1}}{\Gamma_{(\alpha_1)}\Gamma_{(\alpha_2)}}\int_{0}^{\infty} u^{\alpha_1+\alpha_2-1}e^{-u}du\\
&=\frac{\Gamma_{(\alpha_1+\alpha_2)}}{\Gamma_{(\alpha_1)}\Gamma_{(\alpha_2)}}z^{\alpha_1-1} (1-z)^{\alpha_2-1}\\
&= \begin{cases} \frac{1}{\beta (\alpha_1,\alpha_2)}z^{\alpha_1-1}(1-z)^{\alpha_2-1} & 0<z<1\\
                 0 & \text{otherwise}
                 \end{cases}
\end{align}
\end{frame}
\begin{frame}{Solution}
    (b) let z = x + y and w = $\frac{x}{y}$\\
     also, let x_1 =\frac{zw}{1+w} , y_1 = \frac{z}{1+w}\\
     \begin{align}
J &= \begin{vmatrix}
 1 & 1 \\
 \frac{1}{y} & \frac{-x}{y^2}
 \end{vmatrix}
 = - \frac{x+y}{y^2}= - \frac{(1+w)^2}{z}\\
 f_{zw}(z,w) &= \frac{1}{\alpha^{\alpha_1+\alpha_2}\Gamma_{(\alpha_1)}\Gamma_{(\alpha_2)}}\frac{z}{(1+w)^2}\left(\frac{zw}{1+w}\right)^{\alpha_1-1}\left(\frac{z}{1+w}\right)^{\alpha_2-1}e^{-\frac{z}{\alpha}}\\
 &=\frac{1}{\alpha^{\alpha_1+\alpha_2}}\frac{z^{\alpha_1+\alpha_2-1}}{\Gamma_{(\alpha_1)}\Gamma_{(\alpha_2)}}e^{-\frac{z}{\alpha}}\frac{w^{\alpha_1-1}}{(1+w)^{\alpha_1+\alpha_2}}\\
 &=\left(\frac{z^{\alpha_1+\alpha_2-1}}{\alpha^{\alpha_1+\alpha_2}\Gamma_{(\alpha_1+\alpha_2)}}e^{-\frac{z}{\alpha}}\right)\left(\frac{\Gamma_{(\alpha_1+\alpha_2)}}{\Gamma_{(\alpha_1)}\Gamma_{(\alpha_2)}}\frac{w^{\alpha_1-1}}{(1+w)^{\alpha_1+\alpha_2}}\right)\\
 &=f_z(Z)f_w(W) 
 \end{align}
 Thus Z and W are independent random variables
\end{frame}
\begin{frame}{Solution}
(c) let z = x + y and w = $\frac{x}{x+y}$\\
also, let x_1 = zw , y_1 = z-x_1 = z(1-w)\\
\begin{align}
J &= \begin{vmatrix}
 1 & 1 \\
 \frac{y}{(x+y)^2} & \frac{-x}{(x+y)^2}
 \end{vmatrix}
  = \frac{1}{x+y} = \frac{1}{z}\\
   f_{zw}(z,w) &= \frac{z}{\alpha^{\alpha_1+\alpha_2}\Gamma_{(\alpha_1)}\Gamma_{(\alpha_2)}}(zw)^{\alpha_1-1}\{z(1-w)\}^{\alpha_2-1}\\
  f_{zw}(z,w) &= \left(\frac{z}{\alpha^{\alpha_1+\alpha_2}\Gamma_{(\alpha + \beta)}}e^{\frac{-z}{\alpha}}\right)\left(\frac{\Gamma_{(\alpha_1+\alpha_2)}}{\Gamma_{(\alpha_1)} \Gamma_{(\alpha_2)}}w^{\alpha_1-1}(1-w)^{\alpha_2-1}\right)\\
  &= f_Z(z) f_W(w)
 \end{align}
 Thus Z and W are independent random variables.
\end{frame}

\end{document}
